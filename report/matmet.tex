\section{Material and Methods} \label{matmet}


\subsection{Data}


\subsection{Preprocessing}

\gls{eeg} and \gls{emg} signals were resampled from approximatly 200.0Hz to 256.0Hz using
conservative sinc interpolation\citationneeded{Putman}. 
A Frequency of 256.0Hz is convenient since is implies that discrete wavelet decomposition (see feature extraction strategy\TODO{link}) will separate 
frequencies above 4.0Hz from those below 4.0Hz (since $4 = 256.0/{2^6} $).
This frequency is typicaly uses as a cut-off value between theta and delta waves \citationneeded{}.

Annotations labels were resampled at exactly 0.20Hz using nearest neighbour interpolation.

\subsection{Feature exctraction}

In order to avoid edge effects, discrete Wavelet decomposition, using Daubechies wavelet, with four vanishing mooments ("db4"),
was applied \emph{a priory} to the entiere \gls{eeg} and \gls{emg} signals.
This contrasts with applying decomposition for each sucessive temporal segment(\ie{} epoch).
The latter method results in edge effects which would typicaly need to be attenualed by convolving every epoch with a window function (\eg Hamming window).
The decomposition levels were from one to six for the \gls{eeg} and from one to four for the \gls{emg} \TODO{ref figureA}.

Then, each 24h recordings were subdivided in non-overlaping epochs of five seconds.
For both \gls{eeg} and \gls{emg} raw signals, as well as for all their respective wavelet coefficents, an list of \TODO{N} features was computed \TODO{ref figureA, table 2}.
Thus, a total of \TODO{N} features were obtained for each epochs.

A \py{} package, \pr{} (full documentation provided in the appendix), was developed to facilitate feature extraction.

\subsection{Addition of temporal features}
In order to integrate temporal information, new variables, depending on past, and future values of predictors, were added for each epochs.
Given a $\mathbf{X_t}$ is a vector of feature at at time $t$,
A first strategy was to define a new vector of feature $\mathbf{{X'}_t}$ as
\[
\mathbf{{X'}_t} = \{\mathbf{X_{t-\tau}}, ..., \mathbf{X_t}, ..., \mathbf{X_{t+\tau}}\}
\]
where $\tau$ is a discrete lag.

Another strategy was to add features representing the local average $\mathbf{W^n_t}$ of each variable over several window sizes $n$.
\[
\mathbf{{X'}_t} = \{W^{n_1}_t, W^{n_2}_t, ...\}
\]
where
\[
\mathbf{W^n_t} = \frac{1}{2n+1} \sum_{i = t-n}^{t+n}{X_i}
\]
and $n$ is a set of odd integers representing window sizes (\eg $\{1,3,7,15\}$).





\subsection{Stratified Cross Validation}
Generally, success of classification of sleep stages has been assessed by cross-validation \citationneeded{}.
In many studies, it is implied that cross-validation was performed by making $k$ random training subsets 
of the whole data and assessing the model trained by the remaining subsets (\ie{} k-fold crossvalidation).

When working with dense time series (or, for instance, spatial data), it can be suspected that within a statistical block (group),
both features and response variables are very correlated with neighbouring data points.
For instance, the features and label at $t_n$ are expected to be largely similar to the features, and labels at $t_{n+1}$ and $t_{n-1}$.
In statistical terms, $t_{n+1}$ is not independent of $t_{n}$.

If the training sets are drawn completely at random, form a dataset containing multiple long recordings, 
the underlying time series would be made only marginaly sparcer, and the datapoints missing from
a time series could simply be infered from the neighbouring points which remain in the training set.
%~ In this case, the predicting power could be greatly affected by (\ie overlaping short epochs would be very dense).

This could give the impression that the model has a very strong predictive power.
However, the goal of cross validation is to assess how well a predictive model would perform on \emph{new data}. 
The goal of this study, and most similar studies, is to provide a tool to automatically annotate \emph{new recordings}. 
Therefore, it is fairer to perform \emph{stratified cross-validation}, using the different recordings as the stratum levels.
In this study, all cross-validation were performed by training the model with all but one recordings,
and testing it on the recording kept out. This was repeated by successively leaving each recording out.

As shown in \TODO{figure intro}, the prevalence of different states is not balanced. Noticably, \gls{rem} sleep represents only $10\%$ of all epochs.
This means that if all \gls{rem} epoch were misclassified, we could still achieve $90\%$ accuracy.
For this reason, the importance of the variables contributing to segregate the majority classes (wake \emph{vs.} \gls{nrem}) could be overestimated.
Therefore, for the variable elimination\TODO{ref fig} stage and for defining new variables\TODO{fig ref here},
balanced subsampled testing sets (750 epochs of each class) were used.




\subsection{Random forests}
mtry, n-tree, entropy-based confidence.
\subsection{Statistics}

\subsection{Performance assesments}
Comparisons of performance between \pr{} and \pyeeg{} were realised by meseauring the average (between 2 and 100 run, depending on the algorithm) duration of runtime of algorithms on six
normaly distributed ($\mathcal{N}(\mu=0,\sigma=1)$) time series (\ie white noise) of size $n$,
with $n$ between $256 \times{} 5$ and $256 \times{} 30$.
This was repeated five times for each time series.
For computing  Higuchi fractal dimension, $k_{max}$ was set to $2^3$.
For both approximate entropy and sample entropy, the embedding dimension $m$ was set to $2$, and the distance threshold, $r$, to $1.5$.
For fisher information, the embeding dimension and the delay, $\tau$ were set to $3$ and $2$, respectively.
Finally, spectral entropy was computed for the frequency bands bounded by $\{0, 2^0, 2^1, ..., 2^6\}$.



